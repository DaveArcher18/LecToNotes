\documentclass{amsart}
\usepackage[margin=1.5in]{geometry}
\usepackage{amsmath}
\usepackage{amssymb}
\usepackage{amsthm}
\usepackage{hyperref}
\usepackage{tcolorbox}
\usepackage{lastpage}
\usepackage[bbgreekl]{mathbbol}
\usepackage{xcolor}
\usepackage{fancyhdr}
\usepackage{accents}
\DeclareSymbolFontAlphabet{\mathbb}{AMSb}
\DeclareSymbolFontAlphabet{\mathbbl}{bbold}
\input{shortcuts.tex}

\newenvironment{solution}
  {\renewcommand\qedsymbol{$\blacksquare$}
  \begin{proof}[Solution]}
  {\end{proof}}
\renewcommand\qedsymbol{$\blacksquare$}

\usepackage{amsmath, amssymb, tikz, csquotes, multicol, footnote, biblatex, wrapfig, float, mathrsfs, cleveref, enumitem, marginnote, todonotes}
\setlength{\headheight}{40pt}

\theoremstyle{definition}
\newtheorem{theorem}{Theorem}[section]
\newtheorem{lemma}[theorem]{Lemma}
\newtheorem{corollary}[theorem]{Corollary}
\newtheorem{exercise}[theorem]{Exercise}
\newtheorem{question}[theorem]{Question}
\newtheorem{example}[theorem]{Example}
\newtheorem{proposition}[theorem]{Proposition}
\newtheorem{conjecture}[theorem]{Conjecture}
\newtheorem{remark}[theorem]{Remark}
\newtheorem{definition}[theorem]{Definition}
\titleLecture Transcript
\date2025-04-23
\author{Generated Transcript}
\maketitle
\tableofcontents

\begin{document}
\large
\maketitle
\section*{Transcript}


\paragraph*00:00%
 which I think is May 2nd

\paragraph*00:02%
 alright so I want to talk

\paragraph*00:07%
 in this whole course about something

\paragraph*00:09%
 that I call a Maduro homology

\paragraph*00:11%
 and

\paragraph*00:19%
 this is related but different

\paragraph*00:22%
 from the Sabiro stuff I was talking about

\paragraph*00:25%
 last term

\paragraph*00:26%
 so

\paragraph*00:27%
 so last term

\paragraph*00:30%
 the title

\paragraph*00:32%
 was

\paragraph*00:33%
 have you ever

\paragraph*00:34%
 read the

\paragraph*00:34%
 number field

\paragraph*00:35%
 you don't have

\paragraph*00:39%
 to know what

\paragraph*00:40%
 they are

\paragraph*00:40%
 and in some

\paragraph*00:40%
 sense I didn't

\paragraph*00:41%
 really discuss

\paragraph*00:42%
 them all that

\paragraph*00:42%
 much in the

\paragraph*00:43%
 course

\paragraph*00:43%
 and so

\paragraph*00:47%
 it's an appropriate

\paragraph*00:49%
 moment to recall

\paragraph*00:50%
 what they are

\paragraph*00:51%
 and

\paragraph*00:54%
 so are you

\paragraph*00:55%
 writing

\paragraph*00:56%
 are we

\paragraph*00:56%
 Sorry.

\paragraph*01:03%
 Thanks.

\paragraph*01:11%
 A bureau It so hard to remember so there are basically

\paragraph*01:24%
 two sides

\paragraph*01:25%
 to this story

\paragraph*01:28%
 and on the left I will put

\paragraph*01:30%
 some of the things from last term

\paragraph*01:31%
 and the right from this term

\paragraph*01:33%
 so for me the motivation

\paragraph*01:34%
 in defining these things was

\paragraph*01:37%
 some expectations

\paragraph*01:38%
 starting from around

\paragraph*01:45%
 2017 or so

\paragraph*01:46%
 that

\paragraph*01:48%
 there is

\paragraph*01:51%
 some such a neurochromology

\paragraph*01:53%
 and I will use this lecture

\paragraph*01:56%
 to explain

\paragraph*01:56%
 what I mean by this

\paragraph*02:01%
 and these are hearings of number fields

\paragraph*02:03%
 that basically is a zero-dimensional case

\paragraph*02:05%
 if you have varieties, just a bunch of points

\paragraph*02:07%
 in some sense

\paragraph*02:08%
 then you get these are zero rings of number fields

\paragraph*02:11%
 so there is some kind of

\paragraph*02:12%
 abstract

\paragraph*02:14%
 idea of what

\paragraph*02:17%
 these things should be

\paragraph*02:18%
 and last time I was talking about

\paragraph*02:21%
 something much more concrete

\paragraph*02:22%
 so there were certain certain Q related to perturbative confinement theory whatever that is

\paragraph*02:48%
 But these Q-theories were completely explicit, so they were very explicit approach,

\paragraph*02:53%
 which actually defines elements in there.

\paragraph*03:03%
 So the abstract approach produces the ring or the module.

\paragraph*03:06%
 This concrete approach will produce elements in there.

\paragraph*03:10%
 And this abstract side also got started maybe a little later.

\paragraph*03:18%
 someone coming from

\paragraph*03:20%
 computations at Garofalidis and Zagilded

\paragraph*03:23%
 and then somehow we talked to each other

\paragraph*03:25%
 and realized that there is this

\paragraph*03:27%
 interesting common object of interest

\paragraph*03:29%
 and this led to our paper

\paragraph*03:30%
 and at that time I was actually thinking that I didn believe in this expectation anymore

\paragraph*03:51%
 But then they found element in there and just revived the expectation.

\paragraph*03:55%
 And then my student, Ferdinand Wagner, tried to see whether such a thing exists.

\paragraph*04:01%
 And at first, his master's thesis proved no.

\paragraph*04:03%
 But now his PhD thesis proves yes.

\paragraph*04:05%
 So it's a very confusing state of affairs.

\paragraph*04:08%
 and part of the goal of this course is to somehow understand what's going on, although that will still be confusing.

\paragraph*04:16%
 So this half was the last term, and this was the last term.

\paragraph*04:28%
 and

\paragraph*04:31%
 there have been also some developments

\paragraph*04:37%
 that bring these two sides

\paragraph*04:39%
 which so far just give rise to one common object

\paragraph*04:41%
 these sides seem to converge more

\paragraph*04:45%
 and maybe or maybe not I will say something about

\paragraph*04:48%
 what's happening with the relation between these two things

\paragraph*04:51%
 but mostly these are disjoint things

\paragraph*04:53%
 that just happen to overlap in certain objects

\paragraph*04:56%
 and so for this reason you don't


\end{document}